
\input moderncv
% ==================== Main text ====================>
\starttext
\placehead

% self profile
\subject{自我评价}
十余年工作经验,具有扎实的技术功底、刻苦的钻研精神;
处事稳重,工作认真,责任心强,思维缜密,逻辑严谨,学习能力强,善于独立思考,
具有较强的分析解决问题的能力。

\subject{技能}

从开始工作至今从事的多为软件设计开发相关工作,
最近几年由于工作需要,接触了部分 FPGA、 android 相关的设计开发工作。

开发工作主要围绕 Linux 展开,日常使用的是 ArchLinux;
从 bootloader、驱动到上层应用都有所涉及;
涉及交换机、光传输、仪器仪表等领域。

\subject{作品}

\cventry{OpenCL 规范中文版}{
\url[openclspeczh]
}

\cventry{结构化寄存器定义}{
\url[srd]
}

\cventry{libem + zeromsg}{
\url[libem]
}

% work experience
\subject{工作经历}

\cvworkexp{2012 ~ 今}{南京吉隆光纤通信股份有限公司}{研发部}{南京}{南京研发部经理}{
负责整个公司 Linux 技术平台的搭建;

负责新员工技术培训;

主持设计开发下一代熔接机控制系统。

通过电机控制、图像处理、电弧控制等,实现熔接、加热、放电校正、电机测试、灰尘检查、自动调焦等功能。

涉及到的技术:状态机(用于熔接、加热等功能的流程控制)、
socket通信(用于层间通信)、
zeromsg(用于数据的序列化反序列化)、
xml(用于配置文件)、
图像处理(用于光纤定位、缺陷检测、灰尘检查、边缘检测、自动调焦)、
视频编码录制(vp8)等。

FPGA 主要负责电机控制,图像采集、格式转换、图像缩放、图层合成等。

开发环境:xilinx ISE 14.7 + Archlinux + gitlab + poky(yocto) + opencv + libpng + boost + Texlive + Qt5 + maxima(用於精確數學計算)+ android studio

编程语言:verilog、 python、 Makefile、 Con\TeX t、 C++11、 C、 bash、 lua

硬件平台: arm s3c6410 zynq7020

目前arm11平台已基本稳定,用于各系列单纤机、带纤机。
正在开发 xilinx zynq 平台、特种光纤熔接机、android 端 app 等。
}

\cvworkexp{2010 ~ 2012}{浙大网新}{物联网事业部}{杭州}{}{
\startigBase
\item 负责软件部基础设施的预研及建设;
\item 参与软件平台的整体设计,负责底层软件平台的设计开发;
\item 开源软件/项目的研究及使用,相关配套工具的设计开发;
\item 部分招聘和技术培训工作;
\item 作为 SA735 产品的开发代表负责其软件开发和版本发布管理等工作;
\item 参与 SN6500 产品的软件设计开发;
\item 负责数据中心的前期调研工作,包括硬件选型、分布式文件系统、分布式数据库等。
\stopigBase

开发环境主要是 Linux 和 C/C++,其中开发工作包括但不限于:
\startigBase
\item 嵌入式 OS 构建系统 optimus(基于 yocto),
主要是为了解决产品分化的问题,将多个产品统一到同一个平台上,
最开始曾经试图用 LTIB 搭建,但由于 LTIB 自身固有问题效果不大理想,后来改用 yocto。
\item GUI 编译工具:太极图(与 optimus 配套使用,用 wxpython 实现),
主要是以图形化的方式显式软件包之间的依赖关系,并完成自动推导递归编译。
\item 嵌入式设备双系统冗余备份方案的设计开发(包括配套的升级工具)。
\item 软件平台中的进程执行模型、
驱动框架(主要用于实现驱动在用户态和内核态的无缝切换)、
进程及设备管理模块、
用于加载 FPGA 和 CPLD 的软件模块等。
\item 以及 Linux下使用的 mm/md(对 uboot 中相应命令功能进行了扩展,以bash脚本实现)等。
\item 用于生成 UBIFS image 的工具(bash脚本,自动构建时使用)。
\item p2020 和 mpc8315 (freescale / powerpc)相关 bsp 的调试开发。
\item \dots
\stopigBase
}

\cvworkexp{2009 ~ 2010}{诺基亚西门子}{NGMGW}{杭州}{高级软件工程师 & TDM 组长}{
带领 TDM 团队进行 TDM 相关业务的需求分析、相关文档的撰写以及软件设计开发,其中包括 SDH、
PDH 的配置管理(包括告警、性能、APS 等),时钟管理,及相应 redundancy 的处理。

开发环境: Linux、 C/C++、 TNSDL。
}

\cvworkexp{2007 ~ 2009}{UT 斯达康}{光传输事业部}{杭州}{}{
涉及产品:NR40K。主要工作:
\startigBase
\item 低阶交叉芯片的研究及驱动开发;
\item 低阶交叉板、混合光板的开发;
\item 热备份 / SSF / MAPS 等新功能的设计开发;
\item 还有配置管理、业务、算法等模块的开发和维护;
\item 中断管理、定时器管理等。
\stopigBase

开发环境: windows/vxworks、 C/C++。
}

\cvworkexp{2006 ~ 2007}{霓虹灯设计与演示系统}{西安}{}{}{
负责网格编辑模块的设计实现。

包括动作管理器、存档序列化/反序列化、undo / redo 、各种移动/花样效果、调色盘、UI 设计等。

开发环境为 windows + VC。
}

\cvworkexp{2005 ~ 2006}{港湾网络}{研发部交换产品线接入服务组}{北京}{}{
负责BH68系列交换机的软件研发工作。

负责QoS、Acl模块的相关工作。
包括:代码维护,包括上层软件和芯片Qos驱动;
新需求的开发,如NP板对MassACL的支持,QoS、Acl对IPv6的支持等;
新芯片(broadcom / firebolt easyridar)功能测试,
并为使QoS、Acl模块支持新芯片进行了详细设计。
另外还负责DHCP模块的维护优化。

开发环境: windows/vxworks、 C/C++。
}

% education
\subject{教育背景}

\cventry{2004 ~ 2007}{
{\bf 硕士} \hskip 1em {西安电子科技大学} \hskip 1em {西安市} \hskip 1em {计算机科学与技术}
}

\cventry{2000 ~ 2004}{
{\bf 学士} \hskip 1em {西安电子科技大学} \hskip 1em {西安市} \hskip 1em {计算机科学与技术}
}

% language
\subject{语言}

\cventry{英语}{六级}

\stoptext
