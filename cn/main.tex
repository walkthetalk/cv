
\input moderncv
% ==================== Main text ====================>
\starttext
\placehead

% self profile
\subject{自我评价}
十余年工作经验,具有扎实的技术功底、刻苦的钻研精神;
处事稳重,工作认真,责任心强,思维缜密,逻辑严谨,学习能力强,善于独立思考,
具有较强的分析解决问题的能力。

软件开发工作均与 Linux 相关;从 bootloader、内核、驱动到上层应用都有所涉及;
涉及交换机、光传输、仪器仪表等领域。
FPGA 开发工作主要涉及图像处理和电机控制两方面内容。

% work experience
\subject{工作经历}

\cvworkexp{2012 ~ 今}{南京吉隆光纤通信股份有限公司}{研发部}{南京}{南京研发部经理}{


负责整个公司技术平台的搭建

设计开发下一代单纤熔接机、带纤机、双芯皮线机等

设计开发熔接机管控系统

其中开发工作包括但不限于以下内容:
\startigBase
\item bootloader:(uboot+barebox) 包括sd卡、nand、lcd等外设驱动(C + assembly)以及相关移植工作
\item Linux kernel:包括 sd 卡、 nand、 lcd、 i2c、 spi、 cmos、电机等驱动开发(C) ,内核裁剪和移植
\item Linux 开发环境:基于 openembedded,包括环境搭建、 bug调测等 (gcc + gdb + C + python + bash)
\item Linux 操作系统:包括各软件包的补丁、调测、打包,并解决编译时遇到得各种问题,其中有 gcc、Qt4、Qt5、postgresql、lua、potrace、ttf、libvpx、opencv、gsl、libpng 等
\item 嵌入式实时操作系统: freertos
\item FPGA:axi stream 相关generator、broadcaster、VDMA、scaler、filter、采集、blender,图像处理(业务相关)、SPI client、LCD 控制、电机控制、PWM、中断系统等、register file generator (python script)

\item 业务软件:
\startigBase
\item 系统架构设计、开发(epoll + websocket),包括一些系统级功能,如冗余备份、自检、同时支持多种机型、器件,对脚本语言的支持等、密码管理系统、业务状态机(boost/statechart);
\item 图像处理、电机控制等关键模块设计开发,其中图像处理相关的有灰尘检查、实时缺陷检测、光纤定位、光纤识别等。
其中光纤识别已可正常工作,目前在试验通用算法(如SVM)和工具(如tensorflow)进行改善。
\item UI框架 unicorn 的设计开发,基于Qt5 qml
\item 消息通信中间件 jmsg 得设计开发,支持 binary、json、sql 等几种类型
\item 数据库:Temporal Table、Paging、Notify、...
\item 相关工具的集成:lua,主要用于原型阶段修改一些参数、函数。
\stopigBase

\item 工具:
\startigBase
\item 工具: gitlab、tikiwiki、maxima、geogebra、Xilinx ISE、Xilinx Vivado
\item 编程语言:assemble、C、C++11、rust、python、bash、lua、tcl
\item 相关库: Qt、OpenCV、libpng、libgsl、libpqxx、rapidjson、boost、libvpx
\stopigBase
\stopigBase
}

\cvworkexp{2010 ~ 2012}{浙大网新}{物联网事业部}{杭州}{}{
\startigBase
\item 负责软件部基础设施的预研及建设;
\item 参与软件平台的整体设计,负责底层软件平台的设计开发;
\item 开源软件/项目的研究及使用,相关配套工具的设计开发;
\item 部分招聘和技术培训工作;
\item 作为 SA735 产品的开发代表负责其软件开发和版本发布管理等工作;
\item 参与 SN6500 产品的软件设计开发;
\item 负责数据中心的前期调研工作,包括硬件选型、分布式文件系统、分布式数据库等。
\stopigBase

开发环境主要是 Linux 和 C/C++,其中开发工作包括但不限于:
\startigBase
\item 嵌入式 OS 构建系统 optimus(基于 yocto),
主要是为了解决产品分化的问题,将多个产品统一到同一个平台上,
最开始曾经试图用 LTIB 搭建,但由于 LTIB 自身固有问题效果不大理想,后来改用 yocto。
\item GUI 编译工具:太极图(与 optimus 配套使用,用 wxpython 实现),
主要是以图形化的方式显式软件包之间的依赖关系,并完成自动推导递归编译。
\item 嵌入式设备双系统冗余备份方案的设计开发(包括配套的升级工具)。
\item 软件平台中的进程执行模型、
驱动框架(主要用于实现驱动在用户态和内核态的无缝切换)、
进程及设备管理模块、
用于加载 FPGA 和 CPLD 的软件模块等。
\item 以及 Linux下使用的 mm/md(对 uboot 中相应命令功能进行了扩展,以bash脚本实现)等。
\item 用于生成 UBIFS image 的工具(bash脚本,自动构建时使用)。
\item p2020 和 mpc8315 (freescale / powerpc)相关 bsp 的调试开发。
\item \dots
\stopigBase
}

\cvworkexp{2009 ~ 2010}{诺基亚西门子}{NGMGW}{杭州}{高级软件工程师 & TDM 组长}{
带领 TDM 团队进行 TDM 相关业务的需求分析、相关文档的撰写以及软件设计开发,其中包括 SDH、
PDH 的配置管理(包括告警、性能、APS 等),时钟管理,及相应 redundancy 的处理。

开发环境: Linux、 C/C++、 TNSDL。
}

\cvworkexp{2007 ~ 2009}{UT 斯达康}{光传输事业部}{杭州}{}{
涉及产品:NR40K。主要工作:
\startigBase
\item 低阶交叉芯片的研究及驱动开发;
\item 低阶交叉板、混合光板的开发;
\item 热备份 / SSF / MAPS 等新功能的设计开发;
\item 还有配置管理、业务、算法等模块的开发和维护;
\item 中断管理、定时器管理等。
\stopigBase

开发环境: windows/vxworks、 C/C++。
}

\cvworkexp{2005 ~ 2006}{港湾网络}{研发部交换产品线接入服务组}{北京}{}{
负责BH68系列交换机的软件研发工作。

负责QoS、Acl模块的相关工作。
包括:代码维护,包括上层软件和芯片Qos驱动;
新需求的开发,如NP板对MassACL的支持,QoS、Acl对IPv6的支持等;
新芯片(broadcom / firebolt easyridar)功能测试,
并为使QoS、Acl模块支持新芯片进行了详细设计。
另外还负责DHCP模块的维护优化。

开发环境: windows/vxworks、 C/C++。
}

% education
\subject{教育背景}

\cventry{2004 ~ 2007}{
{\bf 硕士} \hskip 1em {西安电子科技大学} \hskip 1em {西安市} \hskip 1em {计算机科学与技术}
}

\cventry{2000 ~ 2004}{
{\bf 学士} \hskip 1em {西安电子科技大学} \hskip 1em {西安市} \hskip 1em {计算机科学与技术}
}

% language
\subject{语言}

\cventry{英语}{CET-6}

\stoptext
